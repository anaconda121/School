\documentclass{article}
\usepackage[utf8]{inputenc}
\usepackage[margin = 1.0in]{geometry}
\usepackage{amsmath}

\title{\textbf{Zinc Lab Report}}
\author{Tanish Tyagi}
\date{September 19, 2022}

\begin{document}

\maketitle

\section{Lab Data}

\subsection{Zinc + HCl Reaction}
Mass of Zinc ($M_{Zn}$): 0.476g \\ 
Mass of Beaker Pre-reaction: 64.289g \\ 
Mass of Beaker Post-reaction: 65.280g

\subsection{Zinc Oxide + HCl Reaction}
Mass of Zinc Oxide ($M_{ZnO_{x}}$): 0.500 g \\
Mass of Beaker Pre-reaction: 65.363g  \\
Mass of Beaker Post-reaction: 66.235g

\section{Analysis Questions}

\begin{enumerate}
\item To find the mass of chlorine ($M_{Cl_{y}}$) in the beaker post-reaction, we can do 65.280g - 64.289g - $M_{Zn}$ = 0.515g. We measured $M_{Zn}$ = 0.476g, so our ratio is $\mathbf{\frac{0.476g}{0.515g} = \frac{476}{515}}$.

\item  We know one of the products for the ZnO + HCl reaction is $ZnCl_{y}$, which has a mass equal to 66.235g - 65.363g = 0.872g. Of this 0.872g, we know that the mass of the Zinc in this compound is $\frac{476}{476 + 515} \cdot 0.872g \approx$ 0.419g. By the law of conservation of mass, the mass of Zinc in $ZnCl_{y}$ (product in this reaction) has to be the same as the mass of Zinc in $ZnO_{x}$ (reactant in this reaction). We also measured the mass of Zinc Oxide---which is composed of Zinc + Oxygen---to be 0.500g. To find the mass of oxygen in $ZnO_{x}$, we can do 0.500g - 0.419g $\approx$ 0.0812g. We can set these masses in a proportion to get $\mathbf{\frac{0.419g}{0.0812g} = \frac{4190}{812}}$. 

\item Assuming formula of compound is ZnCl: \\
The mass ratio will be the same as the ratio we determined in question 1 because ZnCl is composed of one atom of Zinc and Chlorine. \textbf{Ratio is $\mathbf{\frac{476}{515}}$}.

Assuming formula of compound is $ZnCl_{2}$: \\
The Chlorine portion of the ratio will become $\frac{0.515g}{2} \approx$ 0.256g since there are now two Chlorine atoms in the compound, while the Zinc portion of the ratio will stay the same. \textbf{New ratio is $\mathbf{\frac{0.476}{0.256} = \frac{476}{256}}$}.

\item Assuming formula of compound is ZnO: \\
The mass ratio will be the same as the ratio we determined in question 2 because ZnO has one atom of Zinc and Oxygen each. \textbf{Ratio is $\mathbf{\frac{4190}{812}}$}.

Assuming formula of compound is $ZnO_{2}$: \\
The Oxygen portion of the ratio will become $\frac{0.0812g}{2} \approx 0.0406g$, while the Zinc portion of the ratio will remain the same. \textbf{New ratio is $\mathbf{\frac{0.419}{0.0406} = \frac{4190}{406}}$}.

\item The atomic masses of Zinc, Chlorine, and Oxygen are as follows:
\begin{enumerate}
\item Zinc ($\mu_{Zn}$): 65.38amu
\item Chlorine ($\mu_{Cl}$): 35.45amu 
\item Oxygen: ($\mu_{0})$: 15.999amu
\end{enumerate}

We can use these atomic masses to create the following ratios:
\begin{flalign*}
& \frac{\mu_{Zn}}{\mu_{0}} = \frac{65.38amu}{15.999amu} \approx 4.087 & \\
& \frac{\mu_{Zn}}{\mu_{Cl}} = \frac{65.38amu}{35.45amu} \approx 1.844 &
\end{flalign*}

We can now use our ratios derived in the prior questions to identify the correct formulas for Zinc Chloride and Zinc Oxide. The ratio for $ZnCl_2$ is $\approx 1.85$, whereas the ratio for ZnCl is $\approx 0.924$. We can see that the ratio for $ZnCl_{2}$ is extremely close $\frac{\mu_{Zn}}{\mu_{Cl}}$, meaning that the correct formula is $\mathbf{ZnCl_{2}}$. We can now apply similar logic to find the correct formula for Zinc Oxide. The ratio for $ZnO_2$ is $\approx 10.3$, whereas the ratio for $ZnO$ is $\approx 5.16$. ZnO's ratio is closer to $\frac{\mu_{Zn}}{\mu_{0}}$, meaning the correct formula is $\mathbf{ZnO}$.

\end{enumerate}

\section{Discussion Questions}

\begin{enumerate}

\item One source of error is that we did not clean the scoopula tool between scooping Zinc and Zinc Oxide. This means that our breaker used in the ZnO + HCl reaction could contain extra amounts of Zinc. Another source of error is that the HCl might not have fully evaporated when it the beaker was placed on the hot plate. This would mean that there might be excess HCl in our beaker when we measured it post-reaction, which would drive up the mass of $ZnCl_2$. This may have occurred in the ZnO + HCl reaction, as the experimental value for the Zinc to Oxygen ratio is higher than the real value. If the experimental mass of $ZnCl_2$ is higher than the real value, then that means we would calculate a higher mass of Zinc, and therefore a smaller mass of Oxygen. This would increase our experimental ratio for Zinc to Oxygen.

\item The excess HCl was used in this reaction to ensure that all the Zinc reacted. After the reaction occurred, the beaker was placed on the hot plate, where the excess HCl was evaporated. This means, should the lab procedures be executed properly, excess HCl would not be counted in the mass measurement of the beaker post-reaction and other calculations.

\end{enumerate}

\end{document}
