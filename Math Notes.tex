\usepackage{amsmath}

\documentclass{article}
\usepackage[utf8]{inputenc}

\usepackage{hyperref}
\hypersetup{
    colorlinks=true,
    linkcolor=blue,
    filecolor=blue,      
    urlcolor=blue,
    pdfpagemode=FullScreen,
    }

\title{Exeter Math Notes}
\author{Tanish Tyagi}
\date{February 2022}

\begin{document}

\maketitle

\section{Math 421}

\begin{itemize}
    \item Important Derivatives
    
     Page 159 - \url{https://www.exeter.edu/sites/default/files/documents/Math4_5_2021.pdf}
\end{itemize}

\section{Math 431}

\begin{enumerate}

\item \textbf{Derivatives of Inverse Functions}

\item \textbf{Critical Points}

\item \textbf{Local Minimum + Maximum}

\item \textbf{Inflection Points}

\item \textbf{Extreme Value Theorem}

If $f(x)$ is continuous for a range $a$ through $b$, then global maximum or minimum occurs at critical values ($f'(x) = 0$ or $f'(x) = undefined$ or at the endpoints $x = a$ or $x = b$.

Problems: 535, 

\item \textbf{Related Rates}

\item Curvature 

\item \textbf{Euler's Method - Approximating Value of any function given derivative}

Given $f'(x)$ that is not separable and hard to anti-differentiate, you can estimate any $f(x)$ value. 

We can utilize a recursive sequence to store all of our approximations. It is defined as: $y_{n+1} = y_{n} + h(f'(x))$, where $h$ is our step size. To most accurately estimate values, we can make $h$ approach 0.

Example Simulation: \url{https://replit.com/@TanishTyagi123/Eulers-Method-Math-431-550#main.py}

Problems: 550, 

\item Solving Differential Equations 

\item Fundamental Theorem of Calculus 

\item Integral Notation: 
$$ F'(x) = f(x) $$
$$ \int_{a}^{b} x^2 \,dx = F(a) - F(b) $$

\item Riemann Sums - Approximating Integrals

Let's say you want to approximate the integral $\int_{a}^{b} A(x) \,dx$: 

Let's define $\Delta x = \frac{b - a}{n}$

    \begin{itemize}
    \item Left Hand Riemann Sum
$$
\lim_{n\to\infty} \sum_{k=0} ^{n-1} A(k\Delta x)\Delta x 
$$
    \item Right Hand Riemann Sum 
$$
\lim_{n\to\infty} \sum_{k=1} ^{n} A(k\Delta x)\Delta x 
$$

    \item Midpoint Riemann Sum 
    \item Trapezoidal Riemann Sums
% $$
% \lim_{n\to\infty} \sum_{k=1} ^{n} f(kx/n) + f(
% $$
  \end{itemize}
  


\end{enumerate}

\end{document}
