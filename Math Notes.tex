\usepackage{amsmath}

\documentclass{article}
\usepackage[utf8]{inputenc}
\usepackage{mathtools}
\usepackage[thinc]{esdiff}
\usepackage{derivative}
\usepackage{tabularx}

\usepackage{hyperref}
\hypersetup{
    colorlinks=true,
    linkcolor=blue,
    filecolor=blue,      
    urlcolor=blue,
    pdfpagemode=FullScreen,
    }

\title{Phillips Exeter Academy Calculus Notes}
\author{Tanish Tyagi}
\date{February 2022}

\begin{document}

\maketitle

\section{Math 421}

\begin{itemize}
    \item \textbf{Important Derivatives}
    
    Page 159 - \url{https://www.exeter.edu/sites/default/files/documents/Math4_5_2021.pdf}
\end{itemize}

\section{Math 431}

\begin{enumerate}

\item \textbf{Second Derivatives}

Denoted as $f''(x)$ or $\frac{\odif[order={2}]{x}}{\odif[order={2}]{y}}$. 
If $f''(a) > 0$, then the slope of the tangent line increases as x moves from less than $a$ to greater than $a$. Graph will be concave up.

If $f''(a) < 0$, the slope of the tangent line decreases as x moves from less than $a$ to greater than $a$. Graph will be concave down.

\begin{center}
\begin{tabularx}{0.8\textwidth} {
| >{\raggedright\arraybackslash}X 
| >{\centering\arraybackslash}X 
| >{\raggedleft\arraybackslash}X |}
\hline
& $f''(x) < 0$ & $f''(x) > 0$ \\
\hline
$f'(x) < 0$ & function is decreasing at a decreasing rate & function is increasing at an increasing rate\\ 
\hline
$f'(x) > 0$ & function is increasing at a decreasing rate & function is increasing at an increasing rate \\
\hline
\end{tabularx}
\end{center}

\item \textbf{Derivatives of Inverse Functions}

We know that $f(f^{-1}(x)) = x$. We are interested in finding $f'^{-1}x$. If we apply the chain rule, we get $f'(f^{-1}(x)) * f'^{-1}(x) = 1$. Therefore
$f'^{-1}(x) = \frac{1}{f'(f^{-1}(x))}$.

Problems: 547, 554, 

\item \textbf{Critical Points}

When $f'(x) = 0$ or $f'(x) = undefined$. 

\item \textbf{Local Minimum + Maximum}

When $f'(x)$ goes from $ > 0$ to $ < 0$, the point where $f'(x) = 0$ is a local maximum. When $f'(x)$ goes from $ < 0$ to $ > 0$, the point where $f'(x) = 0$ is a local minimum. 

\item \textbf{Inflection Points}

To find inflection points, look for when the second derivative is equal to 0. Inflection point marks the change of concavity ($f''(x) > 0$ to $f''(x) < 0$ or vice versa. Once you find the point where $f''(x) = 0$, make sure to check x-coordinates before and after the point to make sure the concavity actually changes/$f''(x)$ changes signs. 

\item \textbf{Extreme Value Theorem}

If $f(x)$ is continuous for a range $a$ through $b$, then global maximum or minimum occurs at critical values ($f'(x) = 0$ or $f'(x) = undefined$ or at the endpoints $x = a$ or $x = b$.

Problems: 535, 

\item \textbf{Related Rates}

\item \textbf{Curvature} 

\item \textbf{Euler's Method}

Given $f'(x)$ that is not separable and hard to anti-differentiate, you can estimate any $f(x)$ value. 

We can utilize a recursive sequence to store all of our approximations. It is defined as: $y_{n+1} = y_{n} + h(f'(x))$, where $h$ is our step size. To most accurately estimate values, we can make $h$ approach 0.

Example Simulation: \url{https://replit.com/@TanishTyagi123/Eulers-Method-Math-431-550#main.py}

Problems: 550, 

\item Solving Differential Equations 

\item \textbf{Fundamental Theorem of Calculus} 

Part 1: 
$$ F'(x) = f(x) $$
$$ \int_{a}^{b} f(x) \,dx = F(a) - F(b) $$

Problems: 524, 572

\item Riemann Sums - Approximating Integrals

Let's say you want to approximate the integral $\int_{a}^{b} A(x) \,dx$: 

Let's define $\Delta x = \frac{b - a}{n}$

    \begin{enumerate}
    \item Left Hand Riemann Sum
$$
\lim_{n\to\infty} \sum_{k=0} ^{n-1} A(k\Delta x)\Delta x 
$$
    \item Right Hand Riemann Sum 
$$
\lim_{n\to\infty} \sum_{k=1} ^{n} A(k\Delta x)\Delta x 
$$

    \item Midpoint Riemann Sum 
    \item Trapezoidal Riemann Sums
% $$
% \lim_{n\to\infty} \sum_{k=1} ^{n} f(kx/n) + f(
% $$
  \end{enumerate}
  


\end{enumerate}

\end{document}
